\chapter*{Conclusiones}\label{cha:conclusiones}
\addcontentsline{toc}{chapter}{Conclusiones}

En resumen, en este trabajo hemos corroborado que la técnica t-SNE de reducción de la dimensión es apta y ventajosa para la clasificación no supervisada de poses animales durante la ejecución de una tarea motora en el \textit{rotarod} con aceleración. A su vez, vimos que esta clasificación tiene prestaciones que son prometedoras para el estudio del comportamiento animal. En particular, permite automatizar el proceso de clasificación de poses animales, brinda resultados reproducibles y tiene el potencial de cuantificar el comportamiento animal sin la necesidad de que intervenga el sesgo de un experimentador.

Más en detalle, construimos un mapa t-SNE a partir de datos acerca de las posiciones de los ratones y sus partes del cuerpo durante la ejecución de la tarea \textit{rotarod}. Este mapa t-SNE se segmentó utilizando un algoritmo \textit{watershed}, dando origen así a un conjunto finito de clases asociadas a las posiciones de los ratones y sus partes del cuerpo. Se interpretó a este conjunto de $N_{\mathrm{clases}} = 9$ clases como poses específicas (\textit{labels}), que los ratones adoptan durante la ejecución de la tarea motora \textit{rotarod} con aceleración. Más aún, estas poses son representativas a lo largo del conjunto de diferentes ratones con los que se trabajó experimentalmente ($N_{\mathrm{ratones}} = 3$).

La segmentación del mapa t-SNE en regiones discretas fue posible en parte gracias a las restricciones mecánicas que impone la tarea motora estudiada. Esto podría apoyar la hipótesis de emergencia de comportamientos estereotipados producto de restricciones biomecánicas en el movimiento animal.

Además, se corroboró que la duración de las poses clasificadas depende de la velocidad con la que gira el cilindro \textit{rotarod}. Esto es coherente con la intuición de que a mayor velocidad de \textit{rotarod}, mayor es la frecuencia con la que el ratón da pasos sobre el cilindro, por lo que las poses que adopta duran menos tiempo en promedio. Otra característica que se desprende de la clasificación obtenida es que a mayores velocidades de \textit{rotarod} aumenta la probabilidad de transición hacia los \textit{labels} asociados con dar un paso con la patas derecha e izquierda indicando un cambio en el patrón locomotor seleccionado. 

Finalmente, observamos cómo son las transiciones a órdenes mayores entre \textit{labels} y estudiamos así la dinámica de transiciones en secuencias de poses. Un resultado interesante es la aparición de patrones alternantes específicos según la clase de pose en la que nos centremos condicionalmente. También se observaron oscilaciones en el contenido de información de las secuencias de poses condicionales. Estas secuencias de poses podrían corresponder a momentos donde el animal se involucra en la ejecución de un patrón de movimiento específico. 

Cabe mencionar que en este trabajo capturamos la dinámica del movimiento de los ratones esencialmente en las transiciones entre diferentes poses clasificadas. Otro enfoque posible \cite{berman_mapping, berman_3d}, es la utilización de la transformada de \textit{wavelet} Morlet compleja para obtener, a partir las mismas series temporales de las posiciones relativas de los marcadores de las partes del cuerpo de los ratones ya utilizadas, los espectros de frecuencias que componen estas series temporales. De esta manera esperamos obtener información directa acerca de la dinámica del movimiento de los ratones donde los \textit{labels} que se obtendrían se interpretarían como diferentes estrategias o patrones de movimiento, en lugar de poses específicas \cite{berman_mapping, berman_3d}.

Por último, hemos mostrado el potencial de esta clasificación no supervisada de poses para estudiar correlaciones con patrones de actividad neuronal. En particular, mostramos la existencia de neuronas (alrededor de un cuarto de las neuronas registradas) en la región locomotora del mesencéfalo cuya actividad se ve modulada por los eventos de transición entre poses. Complementariamente, podría también estudiarse cómo es la actividad de estas neuronas durante el tiempo en el que ocurre un \textit{label} determinado, además de en los momentos en que ocurren transiciones entre \textit{labels} diferentes. La modulación de la actividad neuronal observada sugeriría que el MLR codifica al menos parte de la información involucrada en el movimiento de los miembros inferiores. Sin embargo, la longitud temporal de la modulación observada es mayor a la duración de las poses individuales, indicando que el MLR participaría de la selección del patrón de movimiento del animal más que a la pose específica en sí.

El paso siguiente para seguir avanzando en el proyecto es aplicar el conocimiento adquirido para estudiar cómo cambian las estrategias de movimiento o las transiciones entre poses durante el aprendizaje de la tarea \textit{rotarod}. Esperamos que los comportamientos clasificados gracias a la técnica t-SNE nos permitan cuantificar detalladamente el desempeño durante el aprendizaje de una tarea motora y así evaluar el repertorio de estrategias de movimiento exitosas o ineficientes durante dicho proceso. Por último, esperamos que la técnica t-SNE de clasificación no supervisada del comportamiento no solo sea capaz de distinguir cambios durante el aprendizaje motor sino también cambios a raíz de impedimentos motores como consecuencia de enfermedades neurodegenerativas. Existe evidencia experimental mostrando una reducción en la latencia a caer en modelos de la enfermedad de Parkinson y de esclerosis lateral amiotrófica \cite{costa_motor_learning, campos_parkinson}, sin embargo estos trabajos utilizan solo la latencia de caída como medida del desempeño del animal. Nosotros postulamos que el estudio detallado del patrón de movimiento durante la ejecución de la tarea nos permitirá la detección temprana de impedimentos motores y la comparación de diferencias entre modelos experimentales de enfermedades neurodegenerativas que ayude a avanzar en la compresión de los mecanismos circuitales subyacentes \cite{alfieri_motor_deficit}.

\thispagestyle{empty}