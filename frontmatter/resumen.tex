%------------------------------------------------------------------------------%
%------------------Resumen-----------------------------------------------------%
%------------------------------------------------------------------------------%
\begin{resumen}
    Los complejos repertorios comportamentales exhibidos por animales pueden ser descritos, con diferentes grados de detalle, como combinaciones de un conjunto finito de movimientos estereotípicos o de estados biofísicos.
    Estas respuestas biofísicas son flexibles, ya que diferentes secuencias de comportamientos pueden usarse para resolver tareas similares, y también son adaptables a cambios en el entorno por medio de mecanismos de aprendizaje.
    Dada esta flexibilidad y adaptabilidad, traducir comportamientos animales complejos a características cuantificables o a categorías bien definidas es una tarea difícil.
    Por un lado, la clasificación manual de comportamientos puede consumir mucho tiempo, requerir la definición de categorías \textit{a priori}, y puede no ser reproducible entre evaluaciones.
    Por otro lado, las categorías comportamentales creadas heurísticamente (por ejemplo, caminar, correr o saltar) suelen ignorar información inherente sobre la variabilidad intra- e inter-animal, típica de los comportamientos no restringidos en entornos naturales.

    Por lo tanto, en este trabajo analizamos diferentes enfoques para cuantificar y evaluar el comportamiento de ratones durante la ejecución de una tarea de aprendizaje motor (rotarod con aceleración). En particular, aplicamos técnicas de aprendizaje automático no supervisado para la clasificación de comportamientos (mapas UMAP con segmentación \textit{watershed}).
    Primeramente, propusimos métricas de rendimiento, alternativas a la latencia a caer, para mostrar diferentes niveles de aptitud física y de aprendizaje de la tarea entre los ratones.
    Luego, usamos mapas UMAP para producir dos posibles representaciones del comportamiento de los ratones, en un espacio latente de dimensión baja.
    Una representación fue construida usando espectros de frecuencia \textit{wavelet} de partes del cuerpo de los ratones y la otra usando características extraídas de sus pasos y poses.
    Finalmente, agrupamos los comportamientos observados en categorías (\textit{labels}), realizando una segmentación \textit{watershed} sobre los mapas, y caracterizamos estos comportamientos.
    De esta manera, dilucidamos la estructura subyacente y la dinámica del comportamiento, mejorando nuestro entendimiento acerca de la ejecución y el aprendizaje de esta tarea. Los comportamientos encontrados separan a los ratones en grupos de rendimiento, y la manera en que estos comportamientos son utilizados se consolida durante el entrenamiento.
\end{resumen}
%------------------------------------------------------------------------------%