%-----------------------------------------------------------------------------%
%------------------Información de Portada-------------------------------------%
%-----------------------------------------------------------------------------%
\titulo{\texorpdfstring{REPRESENTACIONES DEL\\COMPORTAMIENTO ANIMAL EN UNA\\TAREA DE APRENDIZAJE MOTOR}{REPRESENTACIONES DEL COMPORTAMIENTO ANIMAL EN UNA TAREA DE APRENDIZAJE MOTOR}}
\title{\texorpdfstring{REPRESENTATIONS OF\\ANIMAL BEHAVIOR IN A\\MOTOR SKILL LEARNING TASK}{REPRESENTATIONS OF ANIMAL BEHAVIOR IN A MOTOR SKILL LEARNING TASK}}
\orientacion{Orientación: Física en Medicina y Biología}
\autor{\textbf{Álvaro Concha}}
\director{\textbf{Dra. Soledad Espósito}}
\codirector{\textbf{Dr. Damián Hernández}}
\trabajo
{
    TESIS DE MAESTRÍA EN CIENCIAS FÍSICAS
}
\carrera
{
    MAESTRÍA EN CIENCIAS FÍSICAS
}
\grado
{
    MAESTRANDO
}
\laboratorio{\texorpdfstring{Laboratorio de Neurobiología del Movimiento\\Departamento de Física Médica\\Centro Atómico Bariloche}{Laboratorio de Neurobiología del Movimiento Departamento de Física Médica Centro Atómico Bariloche}}
\institucion{\texorpdfstring{Instituto Balseiro\\Comisión Nacional de Energía Atómica\\Universidad Nacional de Cuyo}{Instituto Balseiro Comisión Nacional de Energía Atómica Universidad Nacional de Cuyo}}
\juradoa{\textbf{Dr. Lucas Mongiat}}
\juradob{\textbf{Dr. Luciano Marpegan}}
\juradoc{\textbf{Dr. Sebastián Risau}}
\palabrasclave
{
    rotarod,
    comportamiento animal,
    UMAP,
    aprendizaje automático,
    clasificación no supervisada,
    interpretabilidad de modelos
}
\keywords
{
    rotarod,
    animal bahavior,
    UMAP,
    machine learning,
    unsupervised classification,
    model interpretability
}
\fecha{Febrero de 2022}