%-----------------------------------------------------------------------------%
%------------------Información de Portada-------------------------------------%
%-----------------------------------------------------------------------------%
\titulo{Los mapeos no lineales, ¿qué pueden decirnos acerca de la manera en que un ratón aprende una tarea motora?}
\title{What can non-linear embeddings tell us about the way a mouse learns a motor skill?}
\autor{Álvaro Concha}
\director{Dra. Soledad Espósito}
\codirector{Dr. Damián Hernández}
\trabajo
{
	Tesis
}
\carrera
{
	Carrera de Maestría en Ciencias Físicas
}
\grado
{
    Maestrando
}
\laboratorio{Neurobiología del Movimiento, Departamento de Física Médica,\\Centro Atómico Bariloche -- Comisión Nacional de Energía Atómica}
\jurado
{
	Dr. Lucas Mongiat
}
\palabrasclave
{
	rotarod,
	comportamiento animal,
	UMAP,
	aprendizaje automático,
	clasificación no supervisada,
	interpretabilidad de modelos
}
\keywords
{
	rotarod,
	animal bahavior,
	UMAP,
	machine learning,
	unsupervised classification,
	model interpretability
}
\fecha{Diciembre de 2021}