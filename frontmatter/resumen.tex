%------------------------------------------------------------------------------%
%------------------Resumen-----------------------------------------------------%
%------------------------------------------------------------------------------%
\begin{resumen}
Un supuesto fundamental en la neurociencia del comportamiento es que la actividad animal, durante la ejecución de una tarea definida, puede ser descrita en gran parte mediante un conjunto finito de comportamientos estereotipados \cite{berman_mapping}. A su vez, la información acerca del comportamiento animal puede correlacionarse con registros simultáneos de la actividad neuronal con el fin de estudiar cómo el cerebro codifica diferentes movimientos, cuáles son los circuitos neuronales subyacentes y cómo estos se modifican durante el aprendizaje de nuevas tareas motoras \cite{esposito_defensive,levy_representation}.

Sin embargo, la clasificación en categorías de distintos tipos de comportamientos puede ser un proceso complejo. Por un lado, la extensión de los registros de actividad animal puede ser demasiado grande como para clasificar movimientos manualmente y dicha clasificación manual puede ser poco reproducible entre sujetos. Por otro lado, las categorías creadas heurísticamente (por ejemplo, caminar, correr, saltar) suelen ignorar información inherente a la variabilidad intra- e inter-animal típica de los comportamientos no restringidos \cite{berman_mapping}.

Por lo tanto, en este trabajo se utilizaron técnicas de aprendizaje automático no supervisado para clasificar comportamientos exhibidos por ratones expertos durante la ejecución de una tarea motora. En particular, se utilizó el algoritmo t-SNE para clasificar el comportamiento animal, a partir de un conjunto de datos con información acerca de las posiciones de los ratones y de sus partes del cuerpo mientras caminaban sobre un cilindro que gira a velocidades crecientes (tarea \textit{rotarod} con aceleración) \cite{berman_mapping,vdm_tsne,esposito_rotarod}.

Aplicando dicho algoritmo logramos clasificar el comportamiento en 9 clases individuales que corresponden a poses específicas del ratón, adoptadas mientras realiza la tarea de \textit{rotarod} con aceleración. Estas poses son representativas del conjunto de ratones observados ($N_{\mathrm{ratones}}=3$) y capturan los aspectos comunes de sus movimientos. Además, se estudió la dinámica de transiciones entre poses y la dependencia de las mismas con la velocidad del cilindro \textit{rotarod}. Este análisis nos permitió demostrar que al aumentar la velocidad del cilindro los animales cambian su estrategia motora seleccionando un tipo de locomoción caracterizada por la alternancia entre los miembros inferiores.

Finalmente se utilizó esta clasificación de poses para estudiar posibles correlaciones con patrones de actividad neuronal en la región locomotora del mesencéfalo (MLR) y mostramos la existencia de neuronas cuya actividad está modulada por eventos de transición entre poses. Este trabajo demuestra el beneficio del uso de estrategias no supervisadas para el estudio detallado del comportamiento animal y su utilidad en el estudio funcional de los circuitos neuronales.
\end{resumen}
%------------------------------------------------------------------------------%